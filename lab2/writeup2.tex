% template created by: Russell Haering. arr. Joseph Crop
\documentclass[12pt,letterpaper]{article}
\usepackage{anysize}
\marginsize{2cm}{2cm}{1cm}{1cm}

\begin{document}

\begin{titlepage}
    \vspace*{4cm}
    \begin{flushright}
    {\huge
        ECE 375 Lab 1\\[1cm]
    }
    {\large
        Introduction to AVR Tools
    }
    \end{flushright}
    \begin{flushleft}
    Lab Time: Wednesday 5-7
    \end{flushleft}
    \begin{flushright}
    Sean Rettig
    \vfill
    \rule{5in}{.5mm}\\
    TA Signature
    \end{flushright}

\end{titlepage}

\section{Introduction}

The purpose of the second lab was to acquaint ourselves with the workflow and
concepts associated with programming our AVRs in the C programming language.
We accomplished this by compiling and uploading the given "dance" program to
our AVRs, examining its code to learn how to use the AVR's I/O ports to
manipulate motors, and finally writing our own basic "BumpBot" program using
this new knowledge.

\section{Program Overview}

As described in this lab and the previous lab, this program implements basic
"BumpBot" functionality by moving forward until one of its two front whiskers
is hit, at which point it will reverse for a short time, turn away from the
object it hit (left if the right whisker was hit, or right if the left or both
whiskers were hit), and resume moving forward indefinitely.

Due to the small size of our program, no separate subroutines/functions were
used.  The program first initializes the ports to be used for input (PORTB) and
output (PORTD), allowing the control of the motors and the reading of whisker
data, respectively.  Then, the program instructs the motors to begin moving
forward (by setting PORTD equal to the appropriate bit string) before entering
an infinite loop in which it checks repeatedly if either of the whiskers were
were hit.  This was done by reading PIND and masking all bits except the one
for each whisker by performing a bitwise AND on PIND with a bit string of all
zeroes except in that whisker's bit; the whisker is active if the resulting bit
string is equal to one consisting of all zeroes except that whisker's bit, or
inactive otherwise.  If a whisker is hit, the motors are instructed to move the
bot backwards for 1000 ms, to the side for 1000 ms (once again, left if the
right whisker was hit, or right if the left or both whiskers were hit), and
finally forward again.  The timing of the movement periods was accomplished
simply by sending the motor commands and then simply waiting 1000 ms using
\_delay\_ms(1000) before sending the next command.

\section{Additional Questions}
\begin{enumerate}

    \item This lab required you to begin using new tools for compiling and
        downloading code to your AVR-enabled TekBot using the C language.
        Explain why it is beneficial to write code in a language that can be
        ‘cross compiled.’ Also explain some of the problems of writing in this
        way.

        It can be beneficial to write code in a language that can be
        cross-compiled because it allows programs to execute on various
        different architectures/platforms, often with little or no
        modifications to the code itself.  This is opposed to a language that
        cannot be cross compiled, for which programs often have to be
        significantly modified or sometimes completely rewritten in order to
        run on different hardware than was originally planned.  However,
        cross-compiled languages can also cause problems; because they are
        often higher-level and not specifically designed for the hardware it
        will run on, it can sometimes be difficult, unintuitive, or impossible
        to perform certain actions that a more hardware-specific language could
        do with ease.  Furthermore, because languages that can be cross
        compiled are more generic, there is often a performance penalty to be
        paid as the compiler doesn't always know the intention of the
        programmer and cannot always optimize a program to be as efficient as
        it would be were it written in a more hardware-specific language.

    \item Your program does essentially the same thing as the assembly program
        you downloaded for Lab 1. Compare the size of the output hex files and
        explain the differences in size between them.

        The size of the hex file for the assembly BumpBot program was 485
        bytes, while the C version was 676 bytes.  This is likely related to
        the performance penalties discussed in the previous question; the
        filesize is bigger because it's running more instructions, and it's
        running more instructions because of the overhead of a higher-level,
        more generic language.  The compiler isn't able to optimize the C code
        to be as efficient as straight assembly code.

\end{enumerate}

\section{Conclusion}

The conclusion should sum up the report along with maybe a personal though on
the lab.  For example, in this lab, we were simply required to set up an
AVRStudio4 project with an example program, compile this project and then
download it onto our TekBot bases.  The result of this program allowed the
TekBot to behave in a BumpBot fashion.  The lab was great and allowed us the
time to build the TekBot with the AVR board and learn the software for this
lab.

This lab was fairly basic in its request, simply requiring us to write a
BumpBot program in C for our AVRs.  However, even though the task was simple,
it was ideal because it allowed us to focus on working with the AVR's I/O
interfaces, and as such was a great learning experience that introduced me to
the concepts involved with embedded C programming and using the AVR.

\section{Source Code}

\begin{verbatim}
/*
This code will cause a TekBot connected to the AVR board to
move forward and when it touches an obstacle, it will reverse
and turn away from the obstacle and resume forward motion.

PORT MAP
Port B, Pin 4 -> Output -> Right Motor Enable
Port B, Pin 5 -> Output -> Right Motor Direction
Port B, Pin 7 -> Output -> Left Motor Enable
Port B, Pin 6 -> Output -> Left Motor Direction
Port D, Pin 1 -> Input -> Left Whisker
Port D, Pin 0 -> Input -> Right Whisker
*/

#define F_CPU 16000000
#include <avr/io.h>
#include <util/delay.h>
#include <stdio.h>

int main(void)
{

    DDRB =0b11110000;   //Set up Port B for Input/Output
    PORTB=0b11110000;   //Turn off both motors

    DDRD =0b00000000;   //Set up Port D for Input/Output
    PORTD=0b00000011;   //Set pullup resistors on

    PORTB=0b01100000;   //Move forward
    
    while (1)                           // Loop Forever
        {
            if((PIND | 0b11111100) == 0b11111110){ //Right whisker hit when PIND == 0bxxxxxx10
                PORTB=0b00000000;   //Reverse
                _delay_ms(1000);
                PORTB=0b00100000; //Turn Left
                _delay_ms(1000);
                PORTB=0b01100000;   //Move forward
            }
            else if((PIND | 0b11111100) == 0b11111101 || (PIND | 0b11111100) == 0b11111100){ //Left whisker hit or both whiskers hit at once
                PORTB=0b00000000;   //Reverse
                _delay_ms(1000);
                PORTB=0b01000000; //Turn Right
                _delay_ms(1000);
                PORTB=0b01100000;   //Move forward
            }
        };
    }
\end{verbatim}

\end{document}
