% template created by: Russell Haering. arr. Joseph Crop
\documentclass[12pt,letterpaper]{article}
\usepackage{anysize}
\marginsize{2cm}{2cm}{1cm}{1cm}

\begin{document}

\begin{titlepage}
    \vspace*{4cm}
    \begin{flushright}
    {\huge
        ECE 375 Lab 3\\[1cm]
    }
    {\large
        Data Manipulation and the LCD Display
    }
    \end{flushright}
    \begin{flushleft}
    Lab Time: Wednesday 5-7
    \end{flushleft}
    \begin{flushright}
    Sean Rettig
    David Winkler
    \vfill
    \rule{5in}{.5mm}\\
    TA Signature
    \end{flushright}

\end{titlepage}

\section{Introduction}

The purpose of the third lab was not only to acquaint ourselves with the concepts associated with using the AVR's LCD display, but also programming the AVR in general using assembly, being the first time we have done so.  In particular, we learned about initializing the stack pointer and copying data from program memory to data memory.

\section{Program Overview}

This is a fairly simple program that just loads two predefined strings and displays them on the LCD display.

\subsection{Initialization Routine}

The stack pointer is first initialized so that that stack can be used.  Then the LCD display initialization subroutine is called, and our two strings are transferred from program memory to data memory in two separate loops.  Each loop first loads the program memory address into the Z register and the data memory address that the LCD screen reads from into the Y register.  Then the data is transferred from where Z points to where Y points byte-by-byte using an intermediate register and post-incrementing Z and Y.  This continues until the entire string has been copied.

\subsection{Main Routine}

Since all of our data has already been copied to the proper location in data memory by the time the main routine executes, all that is left to be done is call the "LCDWrite" routine to read the strings from data memory and write them to the LDC Display.  The program then enters an infinite loop so that it doesn't continue executing the rest of program memory (which contains things like our strings).

\section{Additional Questions}
\begin{enumerate}

    \item Take a look at the code you downloaded for today’s lab. Notice the
        lines that begin with .def and .equ followed by some type of
        expression. These are known as pre-compiler directives. Define
        pre-compiler directive. What is the difference between the .def and
        .equ directives? (HINT: see section 5.1 of the AVR Starter Guide given
        on the lab webpage.)

        Pre-compiler directives are instructions specifically for the compiler
        that are executed during the build process (and not on the AVR).
        According to the AVR Starter Guide, a '.def' directive defines a
        symbolic name on a register, while a '.equ' directive sets a symbol
        equal to an expression.

    \item In this lab, you were asked to manipulate data by moving it around in
        memory. In fact, the AVR architecture has two different memories, a
        program memory and data memory. Briefly explain the differences and
        purposes of these memories within your write up.

        The program memory is only writable while flashing the AVR using the
        programmer, and is otherwise read-only, such as when code is executing.
        It holds the contents of programs as well as any constants embedded in
        the program using .DB and/or .DW pre-compiler directives.  The data
        memory is where all other data is stored, and is both readable and
        writeable.  The GPRs and IO registers are also part of the data memory
        addressing scheme.  However, the data memory is only 4KB, while program
        memory is 128KB.

    \item You also learned how to make function calls. Explain how the function
        call works, its connection to the memory stack, and why a RET
        instruction must be called to return from a function. To help you
        understand, comment out the stack pointer initialization in the
        beginning of the code and try running the program. Observe and comment
        on this behavior.

        Function calls begin with the "CALL" or "RCALL" instruction with an
        address to the beginning of a subroutine in program memory as the
        operand.  This then places the current program memory location on the
        stack and sets the program counter (PC) equal to the new address,
        effectively causing the flow of execution to continue at the beginning
        of the called function/subroutine.  Once the subroutine is finished, it
        uses a RET instruction to pop the address from the stack and load it
        back into PC, returning program execution to where it left off before
        the function call.

\end{enumerate}

\section{Conclusion}

In this lab, we learned about the basics of programming our AVRs in assembly,
including the concepts of initializing the stack pointer, embedding strings in
our program data, copying data from program memory to data memory, and
displaying strings on the assembly display.

\section{Source Code}

\begin{verbatim}
\end{verbatim}

\pagebreak

\section{Challenge Source Code}

\begin{verbatim}

;***********************************************************
;*
;*	Enter Name of file here
;*
;*	Enter the description of the program here
;*
;*	This is the skeleton file Lab 3 of ECE 375
;*
;***********************************************************
;*
;*	 Author: Sean Rettig
;*	   Date: 2015-01-21
;*
;***********************************************************

.include "m128def.inc"			; Include definition file

;***********************************************************
;*	Internal Register Definitions and Constants
;***********************************************************
.def	mpr = r16				; Multipurpose register required for LCD Driver


;***********************************************************
;*	Start of Code Segment
;***********************************************************
.cseg							; Beginning of code segment

;-----------------------------------------------------------
; Interrupt Vectors
;-----------------------------------------------------------
.org	$0000					; Beginning of IVs
		rjmp INIT				; Reset interrupt

.org	$0046					; End of Interrupt Vectors

;-----------------------------------------------------------
; Program Initialization
;-----------------------------------------------------------
INIT:							; The initialization routine
		; Initialize Stack Pointer
		LDI R16, LOW(RAMEND) ; Low Byte of End SRAM Address
		OUT SPL, R16 ; Write byte to SPL
		LDI R16, HIGH(RAMEND) ; High Byte of End SRAM Address
		OUT SPH, R16 ; Write byte to SPH

		; Initialize LCD Display
		rcall LCDInit ; Call LCD Init Subroutine

		; Move strings from Program Memory to Data Memory
		ldi ZH, high(STRING_BEG1<<1); Initialize Z-pointer
		ldi ZL, low(STRING_BEG1<<1)

		ldi		YL, low(LCDLn1Addr)
		ldi		YH, high(LCDLn1Addr)

		ldi		r23, 14

LINE1:
		lpm r16, Z+ ; Load constant from Program
		st  Y+, r16
		dec		r23			; Decrement Read Counter
		brne	LINE1
		
		; Move strings from Program Memory to Data Memory
		ldi ZH, high(STRING_BEG2<<1); Initialize Z-pointer
		ldi ZL, low(STRING_BEG2<<1)

		ldi		YL, low(LCDLn2Addr)
		ldi		YH, high(LCDLn2Addr)

		ldi		r23, 12

LINE2:
		lpm r16, Z+ ; Load constant from Program
		st  Y+, r16
		dec		r23			; Decrement Read Counter
		brne	LINE2

		; NOTE that there is no RET or RJMP from INIT, this is
		; because the next instruction executed is the first for
		; the main program

;-----------------------------------------------------------
; Main Program
;-----------------------------------------------------------
MAIN:							; The Main program
		; Display the strings on the LCD Display
		; Move ASCII string to line addresses $0100-$011F
		rcall LCDWrite ; Write string to LCD
		
		rjmp	MAIN			; jump back to main and create an infinite
								; while loop.  Generally, every main program is an
								; infinite while loop, never let the main program
								; just run off

;***********************************************************
;*	Functions and Subroutines
;***********************************************************

;-----------------------------------------------------------
; Func: Template function header
; Desc: Cut and paste this and fill in the info at the 
;		beginning of your functions
;-----------------------------------------------------------
FUNC:							; Begin a function with a label
		; Save variable by pushing them to the stack

		; Execute the function here
		
		; Restore variable by popping them from the stack in reverse order\
		ret						; End a function with RET


;***********************************************************
;*	Stored Program Data
;***********************************************************

;----------------------------------------------------------
; An example of storing a string, note the preceeding and
; appending labels, these help to access the data
;----------------------------------------------------------
STRING_BEG1:
.DB		"Sean and David"		; Storing the string in Program Memory
STRING_BEG2:
.DB		"Hello World!"		; Storing the string in Program Memory
STRING_END:

;***********************************************************
;*	Additional Program Includes
;***********************************************************
.include "LCDDriver.asm"		; Include the LCD Driver
\end{verbatim}

\pagebreak

\section{Challenge Source Code}

\begin{verbatim}
;***********************************************************
;*	Internal Register Definitions and Constants
;***********************************************************
.include "m128def.inc"			; Include the ATMega128 Definition Doc

.def	mpr = r16				; Multi-purpose register defined for LCDDV2
.def	lastchar = r15			; Multi-purpose register defined for LCDDV2
.def	ReadCnt = r23			; Counter used to read data from Program Memory
.def	counter = r4			; Counter used for Bin2ASCII demo
.def	val = r5				; Value to be compared with
.def	TimerCnt = r6			; Counter used for the timer

.equ	CountAddr = $0130		; Address of ASCII counter text

.cseg							; Beginning of code segment

;***********************************************************
;*	Interrupt Vectors
;***********************************************************
.org	$0000					; Beginning of IVs
		rjmp INIT				; Reset interrupt
.org	$001e	
		rjmp TIM0_OVF			; Timer0 Overflow Handler

;***********************************************************
;*	Program Initialization 
;***********************************************************
.org	$0046					; Origin of the Program, after IVs
INIT:							; Initialize Stack Pointer
		ldi		mpr, HIGH(RAMEND)
		out		SPH, mpr
		ldi		mpr, LOW(RAMEND)
		out		SPL, mpr

;		ldi		mpr, 170
;		ldi		XL, 10
;;		ldi		XH, 01
;		rcall   Bin2ASCII

		rcall	LCDInit			; INITIALIZE THE LCD DISPLAY

		; Timer/Counter 0 Initialization
		; Clock Source: System Clock
		; Clock Value: 15.625 kHz
		; Mode: CTC top=OCR0
		; OC0 output: Disconnected
		ldi		mpr, (1<<WGM01|1<<CS02|1<<CS01|1<<CS00)	
		out		TCCR0, mpr		; Set timer prescalar to (clk/64)
		ldi		mpr, $30		; Set timer value
		out		OCR0, mpr		; Set TCNT0
		ldi		mpr, (1<<OCIE0)	; Set Timer 0 OVF interrupt

		; Timer(s)/Counter(s) Interrupt(s) initialization
		out		TIMSK, mpr		; Set TIMSK

		ldi		mpr, 0
		mov		counter, mpr	; Clear the Counter
		clr		TimerCnt		; Clear Timer Counter

		; Write initial string to LCD line 1
		ldi		ZL, low(TXT0<<1); Init variable registers
		ldi		ZH, high(TXT0<<1)
		ldi		YL, low(LCDLn1Addr)
		ldi		YH, high(LCDLn1Addr)
		ldi		ReadCnt, LCDMaxCnt
INIT_LINE1:
		lpm		mpr, Z+			; Read Program memory
		st		Y+, mpr			; Store into memory
		dec		ReadCnt			; Decrement Read Counter
		brne	INIT_LINE1		; Continue untill all data is read
		rcall	LCDWrLn1		; WRITE LINE 1 DATA

		; Write initial string to LCD line 2
		ldi		ZL, low(TXT1<<1); Init variable registers
		ldi		ZH, high(TXT1<<1)
		ldi		YL, low(LCDLn2Addr)
		ldi		YH, high(LCDLn2Addr)
		ldi		ReadCnt, LCDMaxCnt
INIT_LINE2:
		lpm		mpr, Z+			; Read Program memory
		st		Y+, mpr			; Store into memory
		dec		ReadCnt			; Decrement Read Counter
		brne	INIT_LINE2		; Continue untill all data is read
		rcall	LCDWrLn2		; WRITE LINE 1 DATA

		; Activate interrupts
		sei						; Turn on interrupts

;***********************************************************
;*	The main program 
;***********************************************************
MAIN:
		rjmp	MAIN			; Go back to beginning



;***********************************************************
;*	Functions and Subroutines
;***********************************************************

;***********************************************************
;*	Interrupt Service Routines
;***********************************************************

;***********************************************************
;* ISR:		Timer 0 Overflow Interrupt Handler
;* Desc:	This ISR will increment the counter and update
;*			line 1 of the LCD Display to reflect the value
;***********************************************************
TIM0_OVF:
		push	mpr				; Save the mpr
		push	lastchar		; Save the lastchar
		in		mpr, SREG		; Save the SREG
		push	mpr				; 
		
		inc		TimerCnt		; Increment counter
		brne	TIM0_OVF_DONE	; If count is not 0, leave interrupt

		ldi		YL, low(LCDLn1Addr+LCDMaxCnt*2)
		ldi		YH, high(LCDLn1Addr+LCDMaxCnt*2)
		dec		YL
		ld		lastchar, Y
		dec		YL
		ldi		ReadCnt, LCDMaxCnt*2
SHIFT:
		ld		mpr, Y
		inc		YL
		st		Y, mpr
		dec		YL
		dec		YL
		dec		ReadCnt
		brne	SHIFT

		inc		YL
		inc		YL
		st		Y, lastchar

		rcall	LCDWrite

		inc		counter			; Increment the counter for the next round

TIM0_OVF_DONE:
		pop		mpr				; Restore the SREG
		out		SREG, mpr		; 
		pop		lastchar		; Restore the lastchar
		pop		mpr				; Restore the mpr
		reti					; Return from interrupt

;***********************************************************
;*	Data Definitions
;***********************************************************
TXT0:
.DB "Sean and David  "
TXT1:
.DB "Hello World!    "

;***********************************************************
;*	Additional Program Includes
;***********************************************************
.include "LCDDriver.asm"		; Include the LCD Driver
\end{verbatim}

\end{document}
