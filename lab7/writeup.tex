% template created by: Russell Haering. arr. Joseph Crop
\documentclass[12pt,letterpaper]{article}
\usepackage{anysize}
\marginsize{2cm}{2cm}{1cm}{1cm}

\begin{document}

\begin{titlepage}
    \vspace*{4cm}
    \begin{flushright}
    {\huge
        ECE 375 Lab 7\\[1cm]
    }
    {\large
        Timers/Counters
    }
    \end{flushright}
    \begin{flushleft}
    Lab Time: Wednesday 5-7
    \end{flushleft}
    \begin{flushright}
    Sean Rettig
    \vfill
    \rule{5in}{.5mm}\\
    TA Signature
    \end{flushright}

\end{titlepage}

\section{Introduction}

The purpose of this sixth lab was to extend our remote-controlled robot system
from the previous lab by learning how to use timer/counters both for waiting
and adjusting motor speed (or LED brightness).

\section{Program Overview}

\subsection{Remote}

The remote program sends commands to the robot via the USART infrared LED.  For
each command, it first sends a robot ID that is shared between the remote and
the robot.  The robot should only respond to commands preceded by the correct
ID.  The most significant bit of each ID is 0, while the most significant bit
of each command is 1, allowing the types of data to be differentiated.  Since
commands store information in the upper 4 bits of the byte, they are shifted
down so the most significant bit of each command can be a 1; the robot unshifts
the command before running it.

\subsubsection{Initialization Routine}

An interrrupt is set for the debouncing functionality.  Then the stack is
initialized, and the I/O ports are set up; button input, LED output to confirm
button presses, and USART1 output.  Then, USART1 is initialized to run at 2400
baud with 8 data bits and 2 stop bits; the robot is set the same way.  It also
enables transmitting.  The debouncer initialization function is run, and global
interrupts are enabled.

\subsubsection{Main Routine}

The main routine loops over itself, each time checking the button input from
the debouncer to see if a button was pressed, and then running the relevant
command subroutine and resetting the relevant bit in the debouncer result.

\subsubsection{Command Routines}

There is a command routine for each of the 6 possible commands (move forward,
move backward, turn left, turn right, speed up, and slow down).  Each one
simply sends the robot ID byte followed by the command byte.

\subsection{Robot}

The robot waits until it receives a command from the remote, at which point the
command is executed.  If either whisker is hit, it will back up and turn away
in standard BumpBot fashion automatically before it will accept further
commands.

\subsubsection{Initialization Routine}

First, interrupt vectors are set for whisker input and USART input.  The stack
is initialized, and the I/O ports are initialized so the motors and whiskers
can be used (just like in the BumpBot program), as well as USART.  The timers
are then initialized; timers 0 and 2 for controlling the speed of the right and
left motors, and timer 1 for the wait subroutine.  Then, USART1 is initialized
to run at 2400 baud with 8 data bits and 2 stop bits; the remote is set the
same way.  It also enables receiving and receive interrupts.  Interrupts for
the whiskers are also set.

\subsubsection{Main Routine}

The main routine does nothing but loop infinitely, as all actions to be
performed run on interrupts.

\subsubsection{USART\_Receive Routine}

This is the real "main" routine; it runs whenever USART data has been received
and processes it.  It first checks if the data is an ID or a command; if it's
an ID, it checks if it's the correct one, and if so, goes into "accept mode",
meaning that it will run the next command given to it.  If it's given a
command, it will run it only if the robot is already in accept mode, and then
exit accept mode since it had just accepted a command; the next command
received from a remote will be preceded by its own ID byte.  It also includes
logic for ignoring speed up or slow down commands if they would move the speed
outside the range of 0-15, and calls the UpdateTimers subroutine when speed
changes to modify the motor output.

\subsubsection{UpdateTimers Routine}

This routine uses the speed value (0-15) and extrapolates it out to the 0-255
range to be set as the max for timers 0 and 2, causing the motors to change
speed in small increments.

\subsubsection{HitRight Routine}

The HitRight Routine is also triggered by interrupts and runs when the right
whisker is hit.  When this routine runs, it ignores any remote commands given
to it while it moves backward for 1 second and turns left for 1 second.  It
will then continue whatever it what doing before the routine was triggered.  It
accomplishes the timed events by setting movement signals to PORTB and then
calling the Wait routine to wait a short amount of time before setting a new
movement signal.

\subsubsection{HitLeft Routine}

The HitLeft Routine is also triggered by interrupts and runs when the left
whisker is hit.  When this routine runs, it ignores any remote commands given
to it while it moves backward for 1 second and turns right for 1 second.  It
will then continue whatever it what doing before the routine was triggered.  It
accomplishes the timed events by setting movement signals to PORTB and then
calling the Wait routine to wait a short amount of time before setting a new
movement signal.

\subsubsection{Wait Routine}

The Wait routine uses timer/counter 1 to count until 1 second has passed, after
which it returns.

\section{Additional Questions}
\begin{enumerate}

    \item In this lab, you used the Fast PWM mode of both 8-bit timers to
        toggle Pins 7 and 4 of Port B, which is one of many possible ways to
        implement variable speed on a TekBot. Imagine instead that you used
        just one of the 8-bit timers in Normal mode, and every time it
        overflowed, it generated an interrupt.  In that ISR, you manually
        toggled pins 7 and 4 of Port B. What are the advantages and
        disadvantages to this new approach, compared to the original PWM
        approach?

        The primary advantage of this approach is that only one timer is
        required to control the speed of the motors, freeing up the second
        timer for other uses.  However, doing this makes it harder to run the
        motors at different speeds, since they're bothing running off of the
        same timer.

\end{enumerate}

\section{Conclusion}

In this lab, I became familiar with timers, both for waiting/counting and for
generating waveforms that can be used to alter the speed of a motor or the
brightness of an LED.  I also learned a bit about debouncing and polling for
button input, which was especially useful because I wasn't able to get it
working properly for the previous lab (since I was trying to use interrupts).

\section{Source Code}

\subsection{Remote}

\begin{verbatim}
\end{verbatim}

\subsection{Robot}

\begin{verbatim}
\end{verbatim}

\end{document}
