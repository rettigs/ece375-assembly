% template created by: Russell Haering. arr. Joseph Crop
\documentclass[12pt,letterpaper]{article}
\usepackage{anysize}
\usepackage{enumerate}
\marginsize{2cm}{2cm}{1cm}{1cm}

\begin{document}

\begin{titlepage}
    \vspace*{4cm}
    \begin{flushright}
    {\huge
        ECE 375 Lab 1\\[1cm]
    }
    {\large
        Introduction to AVR Tools
    }
    \end{flushright}
    \begin{flushleft}
    Lab Time: Wednesday 5-7
    \end{flushleft}
    \begin{flushright}
    Sean Rettig
    \vfill
    \rule{5in}{.5mm}\\
    TA Signature
    \end{flushright}

\end{titlepage}

\section{Questions}
\begin{enumerate}

    \item Go to the lab webpage and download the template write-up. Read it
        thoroughly and get familiar with the expected format.  Specifically
        look at the included source code. What type of font is used? What size
        is the font? From here on when you include your source code in your lab
        write-up you must adhere to that font type and size.

            It should be noted that this question assumes that the .doc
            template was downloaded rather than the .tex template, which
            provides none of the same information, despite the website implying
            that they were the same.  I originally downloaded just the .tex
            template (and am using it for this document) and was very confused.
            In the .doc template, 8pt Courier New font is used.

    \item Take a look at the code you downloaded for today’s lab. Notice the
        lines that begin with .def and .equ followed by some type of
        expression. These are known as pre-compiler directives. Define
        pre-compiler directive. What is the difference between the .def and
        .equ directives (HINT: see section 5.1 of the AVR Starter Guide given
        on the lab webpage).

            Pre-compiler directives are instructions specifically for the
            compiler that are executed during the build process (and not on the
            AVR).  According to the AVR Starter Guide, a '.def' directive
            defines a symbolic name on a register, while a '.equ' directive
            sets a symbol equal to an expression.

    \item Take another look at the code you downloaded for today’s lab. Read
        the comment that describes the macro definitions. From that explanation
        determine the 8-bit binary value of the following expressions. Note:
        the numbers below are decimal values.

            It should be noted that the explanation in the source code
            incorrectly states that the '$|$' is a logical OR; it is in fact a
            bitwise OR.

            \begin{enumerate}[a.]

                \item $(1<<2)$
                
                    $b00000001<<2$\\
                    $b00000100$

                \item $(2<<1)$
                
                    $b00000010<<1$\\
                    $b00000100$

                \item $(4>>1)$
                    
                    $b00000100>>4$\\
                    $b00000010$

                \item $(1<<4)$
                
                    $b00000001<<4$\\
                    $b00010000$

                \item $(6>>1|1<<6)$
                
                    $b00000110>>1|b00000001<<6$\\
                    $b00000011|b01000000$\\
                    $b01000011$

            \end{enumerate}

\end{enumerate}

\end{document}
