% template created by: Russell Haering. arr. Joseph Crop
\documentclass[12pt,letterpaper]{article}
\usepackage{anysize}
\marginsize{2cm}{2cm}{1cm}{1cm}

\begin{document}

\begin{titlepage}
    \vspace*{4cm}
    \begin{flushright}
    {\huge
        ECE 375 Lab 5\\[1cm]
    }
    {\large
        Simple Interrupts
    }
    \end{flushright}
    \begin{flushleft}
    Lab Time: Wednesday 5-7
    \end{flushleft}
    \begin{flushright}
    Sean Rettig
    \vfill
    \rule{5in}{.5mm}\\
    TA Signature
    \end{flushright}

\end{titlepage}

\section{Introduction}

The purpose of this fifth lab was to familiarize ourselves with the concept of
interrupts and implement the basic "BumpBot" program that we used earlier using
interrupts instead of polling.

\section{Program Overview}

This is a very simple program that implements basic "BumpBot" functionality by
moving forward until one of its two front whiskers is hit, at which point it
will reverse for a short time, turn away from the object it hit (left if the
right whisker was hit, or right if the left or both whiskers were hit), and
resume moving forward indefinitely.

\subsection{Interrupt Vector Initialization}

First, the INT0 and INT1 interrupts are set to activate the HitRight and
HitLeft routines, respectively.

\subsection{Initialization Routine}

All the initialization routine does is initialize the stack pointer, set up the
ports to be used for input (PORTD) and output (PORTB) (allowing the control of
the motors and the reading of whisker data, respectively), and initialize the
external interrupts INT0 and INT1 (in addition to enabling global interrupts in
general).

\subsection{Main Routine}

The main routine just makes the Tekbot move forward and keeps looping over
itself.

\subsection{HitRight Routine}

The HitRight Routine simply moves backward for 1 second, turns left for 1
second, and starts moving forward again.  It accomplishes the timed events by
setting movement signals to PORTB and then calling the Wait routine to wait a
short amount of time before setting a new movement signal.

\subsection{HitLeft Routine}

The HitLeft Routine simply moves backward for 1 second, turns right for 1
second, and starts moving forward again.  It accomplishes the timed events by
setting movement signals to PORTB and then calling the Wait routine to wait a
short amount of time before setting a new movement signal.

\subsection{Wait Routine}

The Wait routine simply runs a long loop during which it does nothing in order
to "wait" a certain amount of time.

\section{Conclusion}

In this lab, I learned about how to initialize and use basic external
interrupts.  Additionally, since this lab involved the reading and writing of a
lot of individual specific bits, I started learning the value of using equate
definitions and bit shifting for addressing particular byte constants (such as
for setting I/O port configurations or the interrupt control registers).

\section{Source Code}

\begin{verbatim}

\end{verbatim}
\end{document}
