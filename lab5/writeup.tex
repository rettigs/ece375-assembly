% template created by: Russell Haering. arr. Joseph Crop
\documentclass[12pt,letterpaper]{article}
\usepackage{anysize}
\marginsize{2cm}{2cm}{1cm}{1cm}

\begin{document}

\begin{titlepage}
    \vspace*{4cm}
    \begin{flushright}
    {\huge
        ECE 375 Lab 5\\[1cm]
    }
    {\large
        Simple Interrupts
    }
    \end{flushright}
    \begin{flushleft}
    Lab Time: Wednesday 5-7
    \end{flushleft}
    \begin{flushright}
    Sean Rettig
    \vfill
    \rule{5in}{.5mm}\\
    TA Signature
    \end{flushright}

\end{titlepage}

\section{Introduction}

The purpose of this fifth lab was to familiarize ourselves with the concept of
interrupts and implement the basic "BumpBot" program that we used earlier using
interrupts instead of polling.

\section{Program Overview}

This is a very simple program that implements basic "BumpBot" functionality by
moving forward until one of its two front whiskers is hit, at which point it
will reverse for a short time, turn away from the object it hit (left if the
right whisker was hit, or right if the left or both whiskers were hit), and
resume moving forward indefinitely.

\subsection{Interrupt Vector Initialization}

First, the INT0 and INT1 interrupts are set to activate the HitRight and
HitLeft routines, respectively.

\subsection{Initialization Routine}

All the initialization routine does is initialize the stack pointer, set up the
ports to be used for input (PORTD) and output (PORTB) (allowing the control of
the motors and the reading of whisker data, respectively), and initialize the
external interrupts INT0 and INT1 (in addition to enabling global interrupts in
general).

\subsection{Main Routine}

The main routine just makes the Tekbot move forward and keeps looping over
itself.

\subsection{HitRight Routine}

The HitRight Routine simply moves backward for 1 second, turns left for 1
second, and starts moving forward again.  It accomplishes the timed events by
setting movement signals to PORTB and then calling the Wait routine to wait a
short amount of time before setting a new movement signal.

\subsection{HitLeft Routine}

The HitLeft Routine simply moves backward for 1 second, turns right for 1
second, and starts moving forward again.  It accomplishes the timed events by
setting movement signals to PORTB and then calling the Wait routine to wait a
short amount of time before setting a new movement signal.

\subsection{Wait Routine}

The Wait routine simply runs a long loop during which it does nothing in order
to "wait" a certain amount of time.

\section{Conclusion}

In this lab, I learned about how to initialize and use basic external
interrupts.  Additionally, since this lab involved the reading and writing of a
lot of individual specific bits, I started learning the value of using equate
definitions and bit shifting for addressing particular byte constants (such as
for setting I/O port configurations or the interrupt control registers).

\section{Source Code}

\begin{verbatim}
;***********************************************************
;*
;*	Lab 5
;*
;*	Enter the description of the program here
;*
;*	This is the skeleton file Lab 5 of ECE 375
;*
;***********************************************************
;*
;*	 Author: Sean Rettig
;*	   Date: 2015-02-04
;*
;***********************************************************

.include "m128def.inc"			; Include definition file

;***********************************************************
;*	Internal Register Definitions and Constants
;***********************************************************
.def	mpr = r16				; Multi-Purpose Register
.def	waitcnt = r17				; Wait Loop Counter
.def	ilcnt = r18				; Inner Loop Counter
.def	olcnt = r19				; Outer Loop Counter

.equ	WTime = 100				; Time to wait in wait loop

.equ	WskrR = 0				; Right Whisker Input Bit
.equ	WskrL = 1				; Left Whisker Input Bit
.equ	EngEnR = 4				; Right Engine Enable Bit
.equ	EngEnL = 7				; Left Engine Enable Bit
.equ	EngDirR = 5				; Right Engine Direction Bit
.equ	EngDirL = 6				; Left Engine Direction Bit

; Using the constants from above, create the movement 
; commands, Forwards, Backwards, Stop, Turn Left, and Turn Right

.equ	MovFwd = (1<<EngDirR|1<<EngDirL)	; Move Forwards Command
.equ	MovBck = $00				; Move Backwards Command
.equ	TurnR = (1<<EngDirL)			; Turn Right Command
.equ	TurnL = (1<<EngDirR)			; Turn Left Command
.equ	Halt = (1<<EngEnR|1<<EngEnL)		; Halt Command

;***********************************************************
;*	Start of Code Segment
;***********************************************************
.cseg							; Beginning of code segment

;-----------------------------------------------------------
; Interrupt Vectors
;-----------------------------------------------------------
.org	$0000					; Beginning of IVs
		rjmp 	INIT			; Reset interrupt

; Set up the interrupt vectors for the interrupts, .i.e
;.org	$002E					; Analog Comparator IV
;		rcall	HandleAC		; Function to handle Interupt request
;		reti					; Return from interrupt
.org	$0002
		rcall	HitRight
		reti
.org	$0004
		rcall	HitLeft
		reti

.org	$0046					; End of Interrupt Vectors

;-----------------------------------------------------------
; Program Initialization
;-----------------------------------------------------------
INIT:	; The initialization routine
		; Initialize Stack Pointer
		LDI R16, LOW(RAMEND) ; Low Byte of End SRAM Address
        OUT SPL, R16 ; Write byte to SPL
        LDI R16, HIGH(RAMEND) ; High Byte of End SRAM Address
        OUT SPH, R16 ; Write byte to SPH

		; Initialize Stack Pointer
		ldi mpr, high(RAMEND)
		out SPH, mpr
		ldi mpr, low(RAMEND)
		out SPL, mpr
		; Initialize Port B for output
		ldi mpr, (1<<EngEnL)|(1<<EngEnR)|(1<<EngDirR)|(1<<EngDirL)
		out DDRB, mpr ; Set the DDR register for Port B
		ldi mpr, $00
		out PORTB, mpr ; Set the default output for Port B
		; Initialize Port D for input
		ldi mpr, (0<<WskrL)|(0<<WskrR)
		out DDRD, mpr ; Set the DDR register for Port D
		ldi mpr, (1<<WskrL)|(1<<WskrR)
		out PORTD, mpr ; Set the Port D to Input with Hi-Z
		; Initialize external interrupts
		; Set the Interrupt Sense Control to level low
		ldi mpr, (0<<ISC01)|(0<<ISC00)|(0<<ISC11)|(0<<ISC10)
		sts EICRA, mpr ; Use sts, EICRA in extended I/O space
		; Set the External Interrupt Mask
		ldi mpr, (1<<INT0)|(1<<INT1)
		out EIMSK, mpr
		; Turn on interrupts
		sei

;-----------------------------------------------------------
; Main Program
;-----------------------------------------------------------
MAIN:	; The Main program

		; Initialize TekBot Foward Movement
		ldi		mpr, MovFwd		; Load Move Forward Command
		out		PORTB, mpr		; Send command to motors

		rjmp	MAIN			; Create an infinite while loop to signify the 
								; end of the program.

;***********************************************************
;*	Functions and Subroutines
;***********************************************************

;----------------------------------------------------------------
; Sub:	HitRight
; Desc:	Handles functionality of the TekBot when the right whisker
;		is triggered.
;----------------------------------------------------------------
HitRight:
		push	mpr			; Save mpr register
		push	waitcnt			; Save wait register
		in		mpr, SREG	; Save program state
		push	mpr			;

		; Move Backwards for a second
		ldi		mpr, MovBck	; Load Move Backwards command
		out		PORTB, mpr	; Send command to port
		ldi		waitcnt, WTime	; Wait for 1 second
		rcall	Wait			; Call wait function

		; Turn left for a second
		ldi		mpr, TurnL	; Load Turn Left Command
		out		PORTB, mpr	; Send command to port
		ldi		waitcnt, WTime	; Wait for 1 second
		rcall	Wait			; Call wait function

		; Move Forward again	
		ldi		mpr, MovFwd	; Load Move Forwards command
		out		PORTB, mpr	; Send command to port

		pop		mpr		; Restore program state
		out		SREG, mpr	;
		pop		waitcnt		; Restore wait register
		pop		mpr		; Restore mpr
		ret				; Return from subroutine

;----------------------------------------------------------------
; Sub:	HitLeft
; Desc:	Handles functionality of the TekBot when the left whisker
;		is triggered.
;----------------------------------------------------------------
HitLeft:
		push	mpr			; Save mpr register
		push	waitcnt			; Save wait register
		in		mpr, SREG	; Save program state
		push	mpr			;

		; Move Backwards for a second
		ldi		mpr, MovBck	; Load Move Backwards command
		out		PORTB, mpr	; Send command to port
		ldi		waitcnt, WTime	; Wait for 1 second
		rcall	Wait			; Call wait function

		; Turn right for a second
		ldi		mpr, TurnR	; Load Turn Left Command
		out		PORTB, mpr	; Send command to port
		ldi		waitcnt, WTime	; Wait for 1 second
		rcall	Wait			; Call wait function

		; Move Forward again	
		ldi		mpr, MovFwd	; Load Move Forwards command
		out		PORTB, mpr	; Send command to port

		pop		mpr		; Restore program state
		out		SREG, mpr	;
		pop		waitcnt		; Restore wait register
		pop		mpr		; Restore mpr
		ret				; Return from subroutine

;----------------------------------------------------------------
; Sub:	Wait
; Desc:	A wait loop that is 16 + 159975*waitcnt cycles or roughly 
;		waitcnt*10ms.  Just initialize wait for the specific amount 
;		of time in 10ms intervals. Here is the general eqaution
;		for the number of clock cycles in the wait loop:
;			((3 * ilcnt + 3) * olcnt + 3) * waitcnt + 13 + call
;----------------------------------------------------------------
Wait:
		push	waitcnt			; Save wait register
		push	ilcnt			; Save ilcnt register
		push	olcnt			; Save olcnt register

Loop:	ldi		olcnt, 224		; load olcnt register
OLoop:	ldi		ilcnt, 237		; load ilcnt register
ILoop:	dec		ilcnt			; decrement ilcnt
		brne	ILoop			; Continue Inner Loop
		dec		olcnt		; decrement olcnt
		brne	OLoop			; Continue Outer Loop
		dec		waitcnt		; Decrement wait 
		brne	Loop			; Continue Wait loop	

		pop		olcnt		; Restore olcnt register
		pop		ilcnt		; Restore ilcnt register
		pop		waitcnt		; Restore wait register
		ret				; Return from subroutine
\end{verbatim}

\section{Challenge Source Code}

\begin{verbatim}
;***********************************************************
;*
;*	Lab 5
;*
;*	Enter the description of the program here
;*
;*	This is the skeleton file Lab 5 of ECE 375
;*
;***********************************************************
;*
;*	 Author: Sean Rettig
;*	   Date: 2015-02-04
;*
;***********************************************************

.include "m128def.inc"			; Include definition file

;***********************************************************
;*	Internal Register Definitions and Constants
;***********************************************************
.def	mpr = r16				; Multi-Purpose Register
.def	waitcnt = r17				; Wait Loop Counter
.def	ilcnt = r18				; Inner Loop Counter
.def	olcnt = r19				; Outer Loop Counter

; if hitlast = 1, right was hit last
; if hitlast = 2, left was hit last
.def    hitlast = r20           ; var for loop detection
; once hitcount = 5, turn more
.def    hitcount = r21          ; var for loop detection
.def	turndouble = r22

.equ	WTime = 100				; Time to wait in wait loop

.equ	WskrR = 0				; Right Whisker Input Bit
.equ	WskrL = 1				; Left Whisker Input Bit
.equ	EngEnR = 4				; Right Engine Enable Bit
.equ	EngEnL = 7				; Left Engine Enable Bit
.equ	EngDirR = 5				; Right Engine Direction Bit
.equ	EngDirL = 6				; Left Engine Direction Bit

; Using the constants from above, create the movement 
; commands, Forwards, Backwards, Stop, Turn Left, and Turn Right

.equ	MovFwd = (1<<EngDirR|1<<EngDirL)	; Move Forwards Command
.equ	MovBck = $00				; Move Backwards Command
.equ	TurnR = (1<<EngDirL)			; Turn Right Command
.equ	TurnL = (1<<EngDirR)			; Turn Left Command
.equ	Halt = (1<<EngEnR|1<<EngEnL)		; Halt Command

;***********************************************************
;*	Start of Code Segment
;***********************************************************
.cseg							; Beginning of code segment

;-----------------------------------------------------------
; Interrupt Vectors
;-----------------------------------------------------------
.org	$0000					; Beginning of IVs
		rjmp 	INIT			; Reset interrupt

; Set up the interrupt vectors for the interrupts, .i.e
;.org	$002E					; Analog Comparator IV
;		rcall	HandleAC		; Function to handle Interupt request
;		reti					; Return from interrupt
.org	$0002
		rcall	HitRight
		reti
.org	$0004
		rcall	HitLeft
		reti

.org	$0046					; End of Interrupt Vectors

;-----------------------------------------------------------
; Program Initialization
;-----------------------------------------------------------
INIT:	; The initialization routine

		; Initialize Stack Pointer
		ldi mpr, high(RAMEND)
		out SPH, mpr
		ldi mpr, low(RAMEND)
		out SPL, mpr

        ; set variables
        ldi hitlast, 0
        ldi hitcount, 0
		ldi turndouble, 0

		; Initialize Port B for output
		ldi mpr, (1<<EngEnL)|(1<<EngEnR)|(1<<EngDirR)|(1<<EngDirL)
		out DDRB, mpr ; Set the DDR register for Port B
		ldi mpr, $00
		out PORTB, mpr ; Set the default output for Port B
		; Initialize Port D for input
		ldi mpr, (0<<WskrL)|(0<<WskrR)
		out DDRD, mpr ; Set the DDR register for Port D
		ldi mpr, (1<<WskrL)|(1<<WskrR)
		out PORTD, mpr ; Set the Port D to Input with Hi-Z
		; Initialize external interrupts
		; Set the Interrupt Sense Control to level low
		ldi mpr, (0<<ISC01)|(0<<ISC00)|(0<<ISC11)|(0<<ISC10)
		sts EICRA, mpr ; Use sts, EICRA in extended I/O space
		; Set the External Interrupt Mask
		ldi mpr, (1<<INT0)|(1<<INT1)
		out EIMSK, mpr
		; Turn on interrupts
		sei

;-----------------------------------------------------------
; Main Program
;-----------------------------------------------------------
MAIN:	; The Main program

		; Initialize TekBot Foward Movement
		ldi		mpr, MovFwd		; Load Move Forward Command
		out		PORTB, mpr		; Send command to motors

		rjmp	MAIN			; Create an infinite while loop to signify the 
								; end of the program.

;***********************************************************
;*	Functions and Subroutines
;***********************************************************

;----------------------------------------------------------------
; Sub:	HitRight
; Desc:	Handles functionality of the TekBot when the right whisker
;		is triggered.
;----------------------------------------------------------------
HitRight:
		push	mpr			; Save mpr register
		push	waitcnt			; Save wait register
		in		mpr, SREG	; Save program state
		push	mpr			;

        cpi     hitlast, 1
        brne    SkipR1      ; If we didn't last hit on the right, skip
        ldi     hitcount, 1 ; If we did, reset the hit count
		ldi		turndouble, 1

SkipR1:

        cpi     hitlast, 2
        brne    SkipR2      ; If we didn't last hit on the left, skip
        inc     hitcount    ; If we did, increment the hit count

SkipR2:

        ldi     hitlast, 1

		; Move Backwards for a second
		ldi		mpr, MovBck	; Load Move Backwards command
		out		PORTB, mpr	; Send command to port
		ldi		waitcnt, WTime	; Wait for 1 second
		rcall	Wait			; Call wait function

		; Turn left for a second
		ldi		mpr, TurnL	; Load Turn Left Command
		out		PORTB, mpr	; Send command to port
		ldi		waitcnt, WTime	; Wait for 1 second
		rcall	Wait			; Call wait function
        cpi     hitcount, 5
        brne    SkipR3
		rcall	Wait			; wait again to turn 180 degrees

SkipR3:

		cpi     turndouble, 1
        brne    SkipR4
		rcall	Wait			; wait again to turn 180 degrees
        ldi     hitcount, 0

SkipR4:

		; Move Forward again	
		ldi		mpr, MovFwd	; Load Move Forwards command
		out		PORTB, mpr	; Send command to port

		ldi		turndouble, 0

		pop		mpr		; Restore program state
		out		SREG, mpr	;
		pop		waitcnt		; Restore wait register
		pop		mpr		; Restore mpr
		ret				; Return from subroutine

;----------------------------------------------------------------
; Sub:	HitLeft
; Desc:	Handles functionality of the TekBot when the left whisker
;		is triggered.
;----------------------------------------------------------------
HitLeft:
		push	mpr			; Save mpr register
		push	waitcnt			; Save wait register
		in		mpr, SREG	; Save program state
		push	mpr			;

        cpi     hitlast, 1
        brne    SkipL1      ; If we didn't last hit on the right, skip
        inc     hitcount    ; If we did, increment the hit count

SkipL1:

        cpi     hitlast, 2
        brne    SkipL2      ; If we didn't last hit on the left, skip
        ldi     hitcount, 1 ; If we did, reset the hit count
		ldi		turndouble, 1

SkipL2:

        ldi     hitlast, 2

		; Move Backwards for a second
		ldi		mpr, MovBck	; Load Move Backwards command
		out		PORTB, mpr	; Send command to port
		ldi		waitcnt, WTime	; Wait for 1 second
		rcall	Wait			; Call wait function

		; Turn right for a second
		ldi		mpr, TurnR	; Load Turn Left Command
		out		PORTB, mpr	; Send command to port
		ldi		waitcnt, WTime	; Wait for 1 second
		rcall	Wait			; Call wait function
        cpi     hitcount, 5
        brne    SkipL3
		cpi     turndouble, 1
        brne    SkipL3
		rcall	Wait			; wait again to turn 180 degrees

SkipL3:

		cpi     turndouble, 1
        brne    SkipL4
		rcall	Wait			; wait again to turn 180 degrees
        ldi     hitcount, 0

SkipL4:

		; Move Forward again	
		ldi		mpr, MovFwd	; Load Move Forwards command
		out		PORTB, mpr	; Send command to port

		ldi		turndouble, 0

		pop		mpr		; Restore program state
		out		SREG, mpr	;
		pop		waitcnt		; Restore wait register
		pop		mpr		; Restore mpr
		ret				; Return from subroutine

;----------------------------------------------------------------
; Sub:	Wait
; Desc:	A wait loop that is 16 + 159975*waitcnt cycles or roughly 
;		waitcnt*10ms.  Just initialize wait for the specific amount 
;		of time in 10ms intervals. Here is the general eqaution
;		for the number of clock cycles in the wait loop:
;			((3 * ilcnt + 3) * olcnt + 3) * waitcnt + 13 + call
;----------------------------------------------------------------
Wait:
		push	waitcnt			; Save wait register
		push	ilcnt			; Save ilcnt register
		push	olcnt			; Save olcnt register

Loop:	ldi		olcnt, 224		; load olcnt register
OLoop:	ldi		ilcnt, 237		; load ilcnt register
ILoop:	dec		ilcnt			; decrement ilcnt
		brne	ILoop			; Continue Inner Loop
		dec		olcnt		; decrement olcnt
		brne	OLoop			; Continue Outer Loop
		dec		waitcnt		; Decrement wait 
		brne	Loop			; Continue Wait loop	

		pop		olcnt		; Restore olcnt register
		pop		ilcnt		; Restore ilcnt register
		pop		waitcnt		; Restore wait register
		ret				; Return from subroutine
\end{verbatim}
\end{document}
